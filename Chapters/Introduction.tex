\chapter{Introduction} \label{introduction}
%\label{sec:Quantum computing}
%As computers get faster, we can solve some algorithms way quicker than before. But there are such %problems and algorithms, that are not trivial, and solving them requires a lot of time and %computing power that even with today's computers would take thousands of years to be solved.
%
%
%
%
%That's where quantum computers come into place as they become useful for handling such problems. %Quantum algorithms can achieve quadratic and even more significant speedups. We know several %applications of this, one is searching in an unordered database or finding a period of a function.
%
%There is a bottleneck in today's quantum computers. It is connected to their runtime possibilities 
%as they can run only a certain amount of time due to coherency, leading to an algorithm's %irrelevant outputs.  
%
Computers are a very useful tool to solving problems, making some tasks automatic. In today's world, you can look at your phone to determine your location on a map. Before you would have to use a more sophisticated approach to find out your location. In today's world you don't have to compute everything by your hand, you can program algorithm and run it on a computer and the computer solves the problem for you. 

But not everything is solvable with our current computers. According to Moore's law \cite{moore:1965}, the number of transistors per microchip doubles approximately every two years. Sometimes, it is referred to Moore's law as doubling computational power every two years, which can be translated as the ability to solve some problem with half the previous time. This is truly helpful when solving algorithms, which require a polynomial time to be solved. But not all algorithms have that property. There are such algorithms with non-polynomial time complexity, meaning that their solving time is determined by, for example, the exponential function. This means that doubling computational power does not really lead to an improvement in time when solving the algorithm for larger input. We will not cover the details such as discussion about \textbf{NP} and \textbf{P} time complexity or the big $O$, $\Theta$ and $\Omega$ notation, which are closely related to time and space complexity. 

This is the place for quantum computers to shine as they have some abilities that classical computers do not have, meaning we can perform algorithms that classical computers are not capable of accomplishing. Quantum computers consist of qubits, which can take advantage of phenomena such as entanglement and superposition. The idea was proposed by Richard Feynman in \cite{feynman2018simulating}, where he questions the possibility of simulating the world with quantum computers. For example, factoring is a hard problem in terms of time complexity. We, as a society, used this to our advantage when creating tools for safe communication \cite{rsa}. In 1994, Peter Shor came with algorithm \cite{shor}, which has exponential speed-up over the classical algorithm, meaning that with a relatively small quantum computer, we would be able to break RSA. Another speed-up is described by Lev K. Grover \cite{grover1996fast}, where he describes an algorithm with quadratic speed-up over the classical algorithm, which will be the topic of this thesis.

On the other hand, we know good algorithms for solving problems, but the algorithm has to have a good machine to run on. This is the bottleneck in quantum computing today. We don't have as good quantum computers as we would like to. We are limited by the size of the computers and decoherence. Simply put, decoherence is connected to qubits and their unexpected behaviour. This behaviour leads to unexpected and wrong results of the computations made by quantum computers. 

One of the leading firms in quantum computing is IBM, their roadmap \cite{roadmap} shows us the future of their approach to making quantum computers reliable and possible to use in solving real problems. An interesting observation is that the number of qubits between 2019 and 2022 increased each year by about three times, which is comparable to the observation about classical computers in \cite{moore:1965}. Also, current computations are made on a single chip, which has limitations in scaling. IBM wants to overcome it by connecting more single chips, so that number of usable qubits will not be determined just by number of qubits on a single chip.

Quantum computing is also very closely related to quantum information science. In 2022 Alain Aspect, John F. Clauser and Anton Zeilinger were awarded the Nobel prize for their work in the field of quantum information \cite{nobel}. Their work is done on the foundations of John S. Bell's work. We will encounter name of a state of a qubit named after him in this thesis.

Quantum computing is a very promising and interesting field with a lot to discover as we are entering the era when we have quantum computers to compute on. It is also seen by the government as a very good investment. 
\endinput