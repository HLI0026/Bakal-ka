\chapter{Basics of quantum computing}

In this section, we are going to describe the basics of quantum computing and briefly sketch the differences between classical computing and quantum computing. We are not going to cover all the basics, but only those needed to understand the next parts of the text.

\section{Qubit}

The name qubit comes from the quantum bit; we are going to use this shortened form in this text. A quantum bit is a unit for carrying information in quantum computers. Just like classical bits, they can be 1 and 0 but unlike bits, they can also be set into superposition, and in this state, they can achieve values 1, 0, and something between, when observed, qubit collaps into either 1 or 0. In this section, we are going to describe what something between means and give it a mathematical representation.


Consider two-dimensional vector space with orthonormal basis states 
$\begin{bmatrix}
     0\\
     1
\end{bmatrix}$
and
$\begin{bmatrix}
     1\\
     0
\end{bmatrix}$
, then, in Dirac notation, we can express 
$\begin{bmatrix}
     0\\
     1
\end{bmatrix}$
as
$|1\rangle$
and
$\begin{bmatrix}
     1\\
     0
\end{bmatrix}$
as
$|0\rangle$. This notation is sometimes called the bra-ket notation, more specifically we mentioned only ket part of the notation. 

Let $| \psi \rangle$ be a  state of a qubit, then we can write it as:
\begin{equation}
    |\psi\rangle = \alpha |0\rangle +\beta |1\rangle
\end{equation}
Where $\alpha, \beta \ \in \mathbb{C}$ and  $|\alpha|^2 + |\beta|^2 = 1$, $|\alpha| ^2$ and $|\beta| ^2$ represent possibilities of measuring qubit in state $| 0 \rangle$, $|1\rangle$ respectively. Therefore, it is possible to rewrite state $| \psi \rangle$ as 
$\begin{bmatrix}
    \alpha \\
    \beta
\end{bmatrix}$. It is important to note that $\lVert | \psi \rangle \rVert$ always has to be 1. 

Say we have two different basis states of our vector space, $\frac{|0\rangle + |1\rangle}{\sqrt{2}}$ and $ \frac{|0\rangle - |1\rangle}{\sqrt{2}}$, then it is possible to refer to them as $|+\rangle$ and $|-\rangle$, where $|+\rangle$ and $|-\rangle$ is also an orthonormal basis.



\section{System of qubits}

Unique qubits alone are unusable for any computation. In this part, we are going to describe connections between qubits with mathematical tools and outline its meaning.

\subsection{Tensor product}
%Let $ a_1, a_2, \ldots, a_n$ and $ b_1, b_2, \ldots, b_m$ be complex numbers.
 Let vector $ |a\rangle =
%
% First Vetor
%
\begin{bmatrix}
     a_1\\
     a_2\\
     \vdots\\
     a_n
\end{bmatrix}$ 
%
%
$\in \mathbb{C}^n $ and 
%
% Second vector
%
$ |b\rangle =
\begin{bmatrix}
     b_1\\
     b_2\\
     \vdots\\
     b_m
\end{bmatrix}$ $\in \mathbb{C}^m$. 
Then we define tensor product of $|a\rangle$ and $|b\rangle$ as: 
\begin{equation}
|a\rangle \otimes |b\rangle = |a\rangle |b\rangle = |ab\rangle =\begin{bmatrix}
     a_1 \cdot b_1\\
     a_1 \cdot b_2\\
     \vdots\\
     a_1 \cdot b_m\\
     a_2 \cdot b_1\\
     a_2 \cdot b_2\\
     \vdots\\
     a_n \cdot b_1\\
     a_n \cdot b_2\\
     \vdots\\
     a_n \cdot b_m\\
\end{bmatrix}
, |ab\rangle \in \mathbb{C}^{n \cdot m} 
\end{equation}

\subsection{Quantum register}

Suppose $|\psi_1\rangle = \alpha_1 |0\rangle +\beta_1 |1\rangle$, $ |\psi_2\rangle = \alpha_2 |0\rangle +\beta_2 |1\rangle$ and $|\psi_3\rangle = \alpha_3 |0\rangle +\beta_3 |1\rangle $ are states of qubits. Then we consider 
\begin{equation} \label{3qubitsystem}
  \begin{aligned}
    |\psi\rangle = |\psi_1 \psi_2 \psi_3\rangle = 
    \alpha_1 \alpha_2 \alpha_3 |000\rangle +
    \alpha_1 \alpha_2 \beta_3 |001\rangle +
    \alpha_1 \beta_2 \alpha_3 |010\rangle +
    \alpha_1 \beta_2 \beta_3 |011\rangle + \\
    \beta_1 \alpha_2 \alpha_3 |100\rangle +
    \beta_1 \alpha_2 \beta_3 |101\rangle +
    \beta_1 \beta_2 \alpha_3 |110\rangle +
    \beta_1 \beta_2 \beta_3 |111\rangle  
    \end{aligned}
\end{equation}
as a quantum register with $ |\alpha_1 \alpha_2 \alpha_3|^2$, $ |\beta_1 \alpha_2 \alpha_3|^2$, \dots , $|\beta_1 \beta_2 \beta_3 |^2$ being probabilities of respective measurements. From simple observation, we are able to see that in kets are numbers which are in binary form. When rewritten in decimal form, they correspond to $[0,2^{n}-1]$, in this case $[0,2^{3}-1]$. Now consider the quantum register $|\psi\rangle$ with n qubits. Then we can expand it as:
\begin{equation}
    |\psi\rangle =  \sum_{i=0}^{2^n-1} a_i |i\rangle 
\end{equation}
Where $i$ in $|i\rangle $ is in binary notation as we explained above.
\begin{equation} \label{prob_sum}
    \sum_{i=0}^{2^n-1} |a_i|^2 = 1 
\end{equation}
\ref{prob_sum} reffers to summing probability of all possible outcomes is equal to one as well as $\lVert | \psi \rangle \rVert = 1$. 

\subsection{Entanglement}
We have covered the composition of multiple qubits in a system with tensor product. Consider $|\psi_1\rangle = \alpha_1 |0\rangle +\beta_1 |1\rangle$, $ |\psi_2\rangle = \alpha_2 |0\rangle +\beta_2 |1\rangle$, and $|\psi \rangle = \frac{|00 \rangle + |11 \rangle }{\sqrt{2}} $. First of all, we see that $| \psi \rangle$ satisfies \ref{prob_sum}. Let us check if we can produce $| \psi \rangle$ from $| \psi_1 \rangle$ and $| \psi_2 \rangle$ with tensor product.
\begin{equation} \label{2qubitsystem}
  \begin{aligned}
    |\widehat{\psi}\rangle = |\psi_1 \psi_2 \rangle = 
    \alpha_1 \alpha_2 |00\rangle +
    \alpha_1 \beta_2 |01\rangle +
    \beta_1 \alpha_2 |10\rangle +
    \beta_1 \beta_2 |11\rangle
    \end{aligned}
\end{equation}
For $|\widehat{\psi}\rangle = |\psi\rangle$ it would have to mean, that in \ref{2qubitsystem} $\alpha_1$ or $\beta_2$ and $\beta_1$ or $\alpha_2$ are equal to zero, but that leads to contradiction. We would not be able to measure $|00\rangle$ or $|11\rangle$ or both the kets, which is false. We know we have the probability $\frac{1}{2}$ to measure each one. 

The system of more qubits is considered entangled if we cannot break it up into states of single qubits, as shown above. There is one more interesting thing in $| \psi \rangle$. Suppose that we measure the first qubit and it is $|0\rangle$, then we know with certainty that the second qubit is also in state $|0\rangle$ and with measuring $|1\rangle$ we know that the second is also $|1\rangle$. 

\section{Quantum gates}

Classical computers use classical gates such as AND, OR, etc. to manipulate the state of bit to compute a result to some problem. We find this in quantum computers and quantum computing as well, where we manipulate the state of qubit, entangle them or swap them in a register. 
Quick reminder, we could represent the state of a qubit by a vector of length two, in \ref{2qubitsystem} a vector of length four was required. Generally speaking, for representation between $n$ qubits, we need a vector of length $2^n$.

To manipulate qubit or qubits we use quantum gates, which are represented mathematically as a matrix. To handle operation on one qubit a matrix must have dimension of $(2,2)$, generally speaking to handle operation on $n$ qubits, the matrix representation have to have dimension of $(2^n ,2^n)$. These matrices also have to be unitary. 

Let $U$ be the complex matrix, $U^{-1}$ its inverse matrix, and $\overline{U}^T$ be the transposed complex conjugate matrix. Unitary matrix satisfy following:
\begin{equation}
U^{-1} = \overline{U}^T 
\end{equation}
Note that $U \cdot \overline{U}^T = I$, where $I$ is identity matrix. 

Let's introduce  few of the basic gates, we are going to use in our circuits.

\begin{center}
\begin{tabular}{ |c|c|c| } 
 \hline
 Name & Symbol & Matrix representation \\ 
 \hline
 Hadamard gate & \Qcircuit @C=1em @R=.7em { & \gate{H} & \qw } & $\begin{bmatrix}
     \frac{1}{\sqrt{2}} & \frac{1}{\sqrt{2}} \\
     \frac{1}{\sqrt{2}} & -\frac{1}{\sqrt{2}}\\
 \end{bmatrix}$ \\ 
 \hline
 
 Not, Pauli-X & \Qcircuit @C=1em @R=.7em { & \gate{X} & \qw } & $\begin{bmatrix}
     0 & 1 \\
     1 & 0\\
 \end{bmatrix}$ \\ 
 \hline
  
  Pauli-Z & \Qcircuit @C=1em @R=.7em { & \gate{H} & \qw } & $\begin{bmatrix}
     1 & 0 \\
     0 & -1\\
 \end{bmatrix}$ \\ 
 \hline

\end{tabular}
\end{center}
Those are only gates acting on a single qubit, we promised more! Let us use them as a building block. Let $G$ be matrix representing some gate acting on one qubit. Controled $G$ means, that $G$ is going to be applied if and only if the controling qubit is in state $|1\rangle$ (works similairly for multi-controlled). $G$'s matrix:
\begin{equation}
    \begin{bmatrix}
        g_{1,1} & g_{1,2} \\
        g_{2,1} & g_{2,2} 
    \end{bmatrix}
\end{equation}
Then we can denote controlled $G$'s matrix with first qubit being controlling one as:

\begin{equation}
    \begin{bmatrix}
        1 & 0 & 0 & 0 \\ 
        0 & 1 & 0 & 0 \\
        0 & 0 & g_{1,1} & g_{1,2} \\
        0 & 0 & g_{2,1} & g_{2,2} 
    \end{bmatrix}
\end{equation}
Step from controlled gate to multi-controlled gate is very simple. Consider $n-1$ controlling qubits and one impacted qubit. Then, the dimension of such matrix is going to be $(2^n ,2^n)$ and it is going to be represented as follows:
\begin{equation}
    \begin{bmatrix}
        1 & 0 & \cdots & 0 & 0 \\ 
        0 & 1 & \cdots & 0 & 0 \\
        \vdots & \vdots & \ddots & \vdots & \vdots \\
        0 & 0 &  \cdots & g_{1,1} & g_{1,2} \\
        0 & 0 &  \cdots & g_{2,1} & g_{2,2} 

    \end{bmatrix}   
\end{equation}
Where all elements are equal to zero, only diagonal are ones and in bottom right corner is our matrix $G$. 

Last gate that is going to be in our circuit is a measurement gate - \Qcircuit @C=.7em @R=.3em {
 & \meter &\qw  \\} . It is unique gate in a way, that it measures qubit, meaning superposition collapses and we get information if qubit is in state $|0\rangle$ or $|1\rangle$. 
